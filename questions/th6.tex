\Large \textbf{مولد گشتاور}
\large \textbf{(۲۵ نمره)}

\normalsize \vspace{0.5cm}

فرض کنید 
$X$
یک متغیر تصادفی حقیقی باشد و مقادیر محدودی می‌تواند بگیرد.
مقدار لگاریتم تابع مولد گشتاور به صورت زیر تعریف می‌شود.
$$\phi(s):=\ln \mathbb{E}[\exp (s X)]=\ln \sum_{x} p(x) \exp (s x)$$
\begin{enumerate}[label=(\alph*)]
	\item
	
نشان دهید که
تابع
$p_{s}(x):=p(x) \exp (s x) \exp (-\phi(s))$ 
یک تابع جرم احتمال است.
	
	\item 
	
اگر
$X_s$
یک متغیر تصادفی باشد و مقادیری که می‌تواند اختیار کند با متغیر تصادفی
$X$
برابر باشد، اما با احتمال‌های
$p_{s}(x)$ (که در بخش قبل معرفی شد)،
نشان دهید که
$\phi^{\prime}(s)=E\left[X_{s}\right]$.

	\item


نشان دهید که
$$\phi^{\prime \prime}(s)=\operatorname{Var}\left(X_{s}\right):=E\left[X_{s}^{2}\right]-E\left[X_{s}\right]^{2}$$
و سپس از روی رابطه‌ی بالا نتیجه بگیرید که 
$\phi^{\prime \prime}(s) \geq 0$
و حالت تساوی فقط وقتی که متغیر تصادفی
$X$
غیر تصادفی
(\lr{deterministic})
باشد رخ می‌دهد و در باقی حالات، نامساوی به صورت اکید است.


\end{enumerate}

